\documentclass[a4paper,11pt,oneside, table]{article}
\usepackage[margin=1in]{geometry}
\usepackage{setspace}
\usepackage{imakeidx}
\usepackage{float}
\usepackage{graphicx}
\usepackage{pdfpages}
\usepackage{csquotes}
\usepackage{caption}
\usepackage{xcolor}
\captionsetup[table]{labelfont=it}
\usepackage{pifont}% http://ctan.org/pkg/pifont

\newcommand{\cmark}{\ding{51}}%
\newcommand{\xmark}{\ding{55}}%

\usepackage{hyperref}
\hypersetup{
    colorlinks=true,
    linkcolor=blue,
    filecolor=magenta,      
    urlcolor=cyan,
    pdftitle={AUSL Piacenza},
    pdfpagemode=FullScreen,
    }

\usepackage{algorithm}
\usepackage{algpseudocode}

\newtheorem{nota}{Nota}

\usepackage[italian]{babel}
\usepackage[
  backend=bibtex,
  style=numeric,
  sorting=ydnt
  ]{biblatex}
\addbibresource{quotes.bib}
\makeindex

\newcommand{\putimage}[4] {
	\begin{figure}[H]
	    \centering
	    \includegraphics[width={#4}\linewidth]{#1}
	    \caption{#2}\label{#3}
	\end{figure}
}

\newcommand{\bigimage}[4] {
	\begin{figure}[H]
	    \centering
	    \includegraphics[height={#4}\textheight]{#1}
	    \caption{#2}\label{#3}
	\end{figure}
}

\newcommand{\putsubimage}[5] {
  \begin{minipage}{{#4}\linewidth}
	    \centering
      \includegraphics[width={#5}\linewidth]{#1}
	    \caption{#2}\label{#3}
	\end{minipage}
}

\newcommand{\putimagecouple}[2] {
  \begin{figure}[!htb]
      \centering
      #1
      \hspace{0.5cm}
      #2
  \end{figure}
}

\newcommand{\putimagequadruple}[4] {
  \begin{figure}[!htb]
      \centering
      #1
      \hspace{0.5cm}
      #2
      \linebreak
      #3
      \hspace{0.5cm}
      #4
  \end{figure}
}

\begin{document}
    \begin{titlepage}
        \noindent
        \begin{minipage}[t]{0.19\textwidth}
            \vspace{-4mm}{\includegraphics[scale=1.15]{logo_unimib.pdf}}
        \end{minipage}
        \begin{minipage}[t]{0.81\textwidth}
        {
                \setstretch{1.42}
                {\textsc{Università degli Studi di Milano - Bicocca}} \\
                \textbf{Scuola di Scienze} \\
                \textbf{Dipartimento di Informatica, Sistemistica e Comunicazione} \\
                \textbf{Corso di laurea magistrale in Informatica} \\
                \par
        }
        \end{minipage}
    	\vspace{40mm}
    	\begin{center}
            {\LARGE{
                    \setstretch{1.2}
                    \textbf{Progetto di Architetture del Software}
                    \par
            }}
        \end{center}
        
        \vspace{50mm}
        
        \vspace{15mm}

        \begin{flushright}
            {\large \textbf{Relazione di:}} \\
            \large{Refolli Francesco} \large{865955}
        \end{flushright}
        
        \vspace{40mm}
        \begin{center}
            {\large{\bf Anno Accademico 2024-2025}}
        \end{center}
        \restoregeometry
    \end{titlepage}

    \printindex
    \tableofcontents
    \renewcommand{\baselinestretch}{1.5}

\section{Introduzione}

\subsection{Traccia}

Si deve progettare e realizzare un sistema di \colorbox{cyan}{monitoraggio remoto} della salute di \colorbox{yellow}{pazienti} e di \colorbox{cyan}{teleriabilitazione} in previsione di un intervento chirurgico.
I pazienti devono essere monitorati per i \colorbox{red}{parametri fisiologici} e rispetto alle \colorbox{red}{attività della vita quotidiana}, inclusa \colorbox{cyan}{l'identificazione} del fatto che il paziente svolge le \colorbox{red}{attività previste dal piano di riabilitazione}.
Se alcuni parametri rilevati superano delle \colorbox{red}{soglie}, il sistema \colorbox{cyan}{deve inviare un allarme al medico curante}, \colorbox{yellow}{il quale deve mettersi in contatto con il paziente attraverso una chiamata}.
Si deve progettare un sistema di telemonitoraggio che:

\begin{enumerate}
  \item acquisisce \textit{in tempo reale} i dati dai sensori secondo tempistiche definite secondo il piano terapeutico del paziente.
  \item supporta il medico nella ridefinizione del piano terapeutico (comporta la variazione delle frequenze di acquisizione dei parametri fisiologici).
  \item deve automaticamente attuare il nuovo piano terapeutico.
  \item controlla se si verificano situazioni anomale (valori dei parametri fisiologici al di fuori delle soglie).
  \item nel caso di situazioni anomale, identifica un medico di turno affinché si rechi fisicamente dal paziente per una visita Nel caso di situazioni di allarme (tipo codice rosso), identifica l'ambulanza pi\`u vicina e l'ospedale pi\`u vicino in cui trasportare il paziente.
  \item notifica il medico di turno identificato inviandogli la cartella sanitaria.
  \item consente al medico di turno di inviare i parametri rilevati, la diagnosi ed altre informazioni relative allo stato di salute del paziente.
  \item acquisisce dalla piattaforma di tracciamento le informazioni delle attività della vita quotidiana svolte.
  \item verifica a fine giornata se nei momenti in cui doveva svolgere degli esercizi di riabilitazione, il paziente li ha svolto realmente.
\end{enumerate}

\subsection{Legenda}

\begin{itemize}
  \item \colorbox{cyan}{azioni}
  \item \colorbox{yellow}{attori}
  \item \colorbox{red}{dati}
\end{itemize}

\subsection{Todo}

\begin{itemize}
  \item Finire l'architettura logica
  \item Inserire le note su frequenze, tempo di completamento, locazione e complessit\`a nelle attivit\`a.
\end{itemize}

\section{Architettura del Problema}

\subsection{Assunzioni}

\paragraph{I medici}
\begin{itemize}
  \item La nomenclatura dei medici \`e la seguente: il medico curante \`e definito \textbf{Medico di Base} (o MB), quello di turno \`e il \textbf{Medico di Turno} (o MT).
  \item Il Medico di Turno \`e inteso come personale della Guardia Medica o dell'Ospedale in mobilit\`a che pu\`o gestire le anomalie.
  \item Il Medico di Turno, l'Ambulanza e l'Ospedale possono trovarsi al momento della chiamata in localit\`a diverse.
  \item Il Medico di Turno ha un elenco di destinatari a cui inviare i dati (specialisti, centralina AUSL) che non comprende il Medico di Base in quanto si assume che egli abbia accesso alle informazioni del suo paziente (aggiornate dal sistema).
  \item Il Medico di Turno termina la gestione dell'anomalia manualmente e pu\`o aggiungere una diagnosi alla cartella clinica del paziente.
\end{itemize}

\paragraph{Il monitoraggio}
\begin{itemize}
  \item Il Piano Terapeutico definisce il monitoraggio di un parametro assieme alle sue soglie (min, max) di normalit\`a e al codice anomalia da assegnare in caso di superamento delle soglie.
  \item Con "superamento della soglia" si intende un valore vuori dal range (min, max).
  \item In caso di pi\`u soglie superate, il codice assegnato all'anomalia \`e quello pi\`u grave tra le violazioni.
  \item Un valore fuori soglia fa scattare l'anomalia se e solo se nel controllo precedente non era gi\`a stata superata.
  \item I dati rilevati dai sensori dei parametri sanitari sono salvati (cifrati) nel sistema per permettere al Medico di Turno e al Medico di Base di accedere allo storico recente (< 1 Mese) del paziente.
  \item Il sistema effettua i controlli solo quando i valori attualizzati si rendono disponibili (la frequenza di rilevazione di ciascuno \`e definita nel Piano Terapeutico).
\end{itemize}

\paragraph{Le anomalie}
\begin{itemize}
  \item Se una nuova violazione delle soglie avviene mentre la gestione di un'anomalia \`e in atto, il codice dell'anomalia \`e aggiornato mantenendo il codice pi\`u grave.
  \item Se il codice dell'anomalia \`e promosso da \textbf{non-grave} a \textbf{grave} (locuzione "Anomalia aggravata") allora viene attivata la sotto-procedura di gestione delle emergenze prevista (allerta di Ospedale e Ambulanza).
  \item L'anomalia gestita \`e salvata nello storico delle anomalia a supporto del Medico di Base (e del Medico di Turno nelle future anomalie).
\end{itemize}

\subsection{Casi d'Uso}

Sono stati identificati i seguenti casi d'uso:
\begin{itemize}
  \item \textbf{U1}: Acquisizione Parametro Sanitario
  \item \textbf{U2}: Acquisizione ADL
  \item \textbf{U3}: Acquisizione Attivita' Motoria
  \item \textbf{U4}: Invia Dati Paziente
  \item \textbf{U5}: Monitoraggio Anomalie Sanitarie
  \item \textbf{U6}: Gestione Anomalia
  \item \textbf{U7}: Controllo Aderenza Piano Terapeutico
  \item \textbf{U8}: Definizione Piano Terapeutico
  \item \textbf{U9}: Termina Gestione Anomalia
\end{itemize}

\putimage{images/Diagramma dei Casi d'Uso.png}{Diagramma dei Casi d'Uso}{png:diagramma-dei-casi-duso}{1}

\subsection{Dati del Problema}

\putimage{images/Diagramma dei Dati.png}{Diagramma dei Dati}{png:diagramma-dei-dati}{1}

\subsection{Attivita' dei Casi d'Uso}

I casi d'uso a sua volta sono stati divisi nelle seguenti attivit\`a. I diagrammi contengono alcune note:

\begin{itemize}
  \item In Arancio sono indicate delle informazioni di tipo funzionale.
  \item In Lilla sono indicate delle informazioni sui delay tra le attivit\`a.
  \item In Blu Scuro sono indicate delle informazioni sulla complessit\`a delle attivit\`a (salvo esplicit\`a indicazione, le attivit\`a del sistema hanno bassa complessit\`a).
  \item In Verde sono indicate delle informazioni circa le frequenze massime previste per i flussi e le attivit\`a.
\end{itemize}

\subsubsection{Acquisizione Parametro Sanitario}

Le frequenze sono alte per via del fatto che solitamente si tratteranno di parametri vitali e quindi \`e importante che il sistema possa ricevere molti dati in tempo reale e possa immagazzinarli per future elaborazioni.

\putimage{images/Diagramma delle Attivita'/Acquisizione Parametro Sanitario.png}{Acquisizione Parametro Sanitario}{png:act:acquisizione-parametro-sanitario}{1}

\subsubsection{Acquisizione ADL}

Le frequenze sono pi\`u basse rispetto alla rilevazione di parametri sanitari perch\`e non rappresentano un pericolo immediato per il paziente ma solo una violazione del suo piano terapeutico.

\putimage{images/Diagramma delle Attivita'/Acquisizione ADL.png}{Acquisizione ADL}{png:act:acquisizione-adl}{1}

\subsubsection{Acquisizione Attivita' Motoria}

Le frequenze sono pi\`u basse rispetto alla rilevazione di parametri sanitari perch\`e non rappresentano un pericolo immediato per il paziente ma solo una violazione del suo piano terapeutico.

\putimage{images/Diagramma delle Attivita'/Acquisizione Attivita' Motoria.png}{Acquisizione Attivit\`a Motoria}{png:act:acquisizione-attivita-motoria}{1}

\subsubsection{Invia Dati Paziente}

Si suppone che sia necessario inviare i dati di un paziente circa al massimo una volta al giorno.

\putimage{images/Diagramma delle Attivita'/Invia Dati Paziente.png}{Invia Dati Paziente}{png:act:invia-dati-paziente}{1}

\subsubsection{Monitoraggio Anomalie Sanitarie}

Se le rilevazioni di parametri sono molto frequenti allora anche la loro verifica deve essere altrettanto frequente.
Si stima che un paziente particolarmente problematico possa avere al massimo 10 crisi l'ora (da l\`i le frequenze di classificazione).
Una volta rilevata un'anomalia, la sua classificazione deve essere molto rapida perch\`e alcune patologie sono tempo dipendenti.

\putimage{images/Diagramma delle Attivita'/Monitoraggio Anomalie Sanitarie.png}{Monitoraggio Anomalie Sanitarie}{png:act:monitoraggio-anomalie-sanitarie}{1}

\subsubsection{Gestione Anomalia}

Vale il discorso di prima sulla stima del numero massimo di anomalie per un paziente. Inoltre l'identificazione delle risorse per gestirla (medico di turno, ambulanza e ospedale) deve essere relativamente rapida per la possibile presenza di patologie tempodipendenti. Siccome l'identificazione pu\`o concernere l'accesso di dati in tempo reale per l'allocazione delle risorse e una comunicazione con la centrale operativa, il delay \`e pi\`u rilassato \footnote{\`E un sistema distribuito, dopotutto} e la sua complessit\`a \`e un filo pi\`u complessa rispetto alle altre.

\putimage{images/Diagramma delle Attivita'/Gestione Anomalia.png}{Gestione Anomalia}{png:act:gestione-anomalia}{1}

\subsubsection{Controllo Aderenza Piano Terapeutico}

L'aderenza al piano viene controllata una volta al giorno a mezzanotte \textit{Ovvero allo scadere del giorno} e non essendo un'operazione complessa, ma nemmeno eccessivamente prioritaria, pu\`o essere completata in una decina di min.

\putimage{images/Diagramma delle Attivita'/Controllo Aderenza Piano Terapeutico.png}{Controllo Aderenza Piano Terapeutico}{png:act:controllo-aderenza-piano-terapeutico}{1}

\subsubsection{Definizione Piano Terapeutico}

La ridefinizione del piano terapeutico si stima che possa avvenire al massimo una volta al giorno e deve essere completata entro il tempo di una normale sessione interattiva.

\putimage{images/Diagramma delle Attivita'/Definizione Piano Terapeutico.png}{Definizione Piano Terapeutico}{png:act:definizione-piano-terapeutico}{1}

\subsubsection{Termina Gestione Anomalia}

La terminazione dell'anomalia gestita si stima che possa avvenire massimo un paio di volte al giorno e, prevedendo un'opzionale dichiarazione di diagnosi, deve essere completata entro il tempo di una normale sessione interattiva.

\putimage{images/Diagramma delle Attivita'/Termina Gestione Anomalia.png}{Termina Gestione Anomalia}{png:act:termina-gestione-anomalia}{1}

\section{Architettura Logica}

Per l'architettura logica sono state realizzate due opzioni di partizionamento, quindi valutate in termini di metriche AL ai fini di scegliere la pi\`u vantaggiosa.

\subsection{Partizionamento livellato}

Il sistema \`e diviso in 4 moduli \footnote{Il \textit{modulo} \`e il componente logico, ho separato la terminologia per favorire la comprensione}:
\begin{itemize}
  \item \textbf{C4}: Interazione Guidata Umana.
  \item \textbf{C3}: Gestione Emergenze.
  \item \textbf{C2}: Elaborazione Dati.
  \item \textbf{C1}: Acquisizione Dati.
\end{itemize}

\bigimage{images/Diagramma delle Attivita'/Partizione Logica Livellata.png}{Partizione Logica Livellata}{png:act:partizione-logica-livellata}{1}

Il partizionamento del sistema \`e legato ad un grado di astrazione rispetto al tipo di attivit\`a che sono eseguite e dal grado di intervento umano. Quindi abbiamo una segregazione di tutte le funzioni che richiedono di elaborare dati in un apposito modulo, ivi quelle che acquisiscono dati, ivi la gestione delle emergenze e cos\`i via.

\begin{center}
  \resizebox{\linewidth}{!}{
    \begin{tabular}{|l | r | c|}
      \hline
      Dimensione & Valore & Motivazione \\
      \hline
      Complexity & 30 & La maggior parte di moduli incorpora attivit\`a con livello omogeneo di complessit\`a \\
      Frequency & 60 & Alcuni moduli hanno attivit\`a con frequenze molto diverse, ma alcuni sono omogenei \\
      Delay & 70 & Alcuni moduli hanno attivit\`a con delay molto diversi tra loro \\
      Location & 0 & Le attivit\`a nei moduli eseguono potenzialmente sugli stessi nodi \\
      Extra flows & 40 & Quasi tutti i moduli scambiano informazioni con sistemi esterni (la Centrale, i rilevatori IoT, ... etc) \\
      Intra flows & 40 & Generalmente i moduli si scambiano pochi dati, ma i moduli \textit{Acquisizione Dati} e \textit{Elaborazione Dati} sono molto accoppiati \\
      Sharing & 0 & I moduli non condividono dati non in streaming \\
      Control flows & 10 & L'unica interazione rilevante \`e il trigger delle anomalie in \textit{Elaborazione Dati} verso la \textit{Gestione Emergenze} \\
      \hline
    \end{tabular}
  }
\end{center}

\subsection{Partizionamento settoriale}

Il sistema \`e diviso in 4 moduli:
\begin{itemize}
  \item \textbf{C1}: Rilevazione di dati in tempo reale.
  \item \textbf{C2}: Rilevazione e gestione delle anomalie (dall'inizio alla fine).
  \item \textbf{C3}: Gestione e controllo aderenza ai piani terapeutici.
  \item \textbf{C4}: L'interfaccia "medicale" che permette ai medici di visualizzare e inviare dati.
\end{itemize}

\bigimage{images/Diagramma delle Attivita'/Partizione Logica Settoriale.png}{Partizione Logica Settoriale}{png:act:partizione-logica-settoriale}{1}

Il partizionamento del sistema \`e legato alla frequenza delle attivit\`a (C1) e al tipo di dato che viene processato (C3, C4) e alla comunicazione interna-esterna nella coordinazione del sistema per raggiungere un obiettivo complesso (C2, la gestione di una anomalia dall'inizio alla fine dell'emergenza).
Di seguito una valutazione della partizione nelle 9 dimensioni definite:

\begin{center}
  \resizebox{\linewidth}{!}{
    \begin{tabular}{|l | r | c|}
      \hline
      Dimensione & Valore & Motivazione \\
      \hline
      Complexity & 30 & La maggior parte di moduli incorpora attivit\`a con livello omogeneo di complessit\`a \\
      Frequency & 60 & Alcuni moduli hanno attivit\`a con frequenze molto diverse, ma alcuni sono omogenei \\
      Delay & 60 & Alcuni moduli hanno attivit\`a con delay molto diversi tra loro, ma alcuni sono abbastanza omogenei come ordine di grandezza \\
      Location & 0 & Le attivit\`a nei moduli eseguono potenzialmente sugli stessi nodi \\
      Extra flows & 40 & Quasi tutti i moduli scambiano informazioni con sistemi esterni (la Centrale, i rilevatori IoT, ... etc) \\
      Intra flows & 30 & I moduli si scambiano pochi dati, e sempre al massimo con un solo modulo \\
      Sharing & 0 & I moduli non condividono dati non in streaming \\
      Control flows & 0 & I moduli tra loro non interagiscono \\
      \hline
    \end{tabular}
  }
\end{center}

\subsection{Confronto e partizionamento scelto}

Dalle metriche (riportate nelle Figure \ref{png:metriche-livellata} e \ref{png:metriche-settoriale}) risulta che il partizionamento migliore sia quello settoriale.

\putimagecouple
{\putsubimage{images/metriche-livellata.png}{Partizione per Livelli}{png:metriche-livellata}{0.45}{1}}
{\putsubimage{images/metriche-settoriale.png}{Partizione Settoriale}{png:metriche-settoriale}{0.45}{1}}

\section{Architettura Concreta}

\end{document}
